% pdflatex --shell-escape lecture9.tex && pdflatex --shell-escape lecture9.tex  
\input{../KU-Beamer-Template/style/template.tex}
\usepackage{../KU-Beamer-Template/style/koc} 
\usepackage{minted}
\usepackage{upquote}
\usepackage{graphicx}

\title{KOLT Python}
\subtitle{Python Modules and Third-Party Packages} 
\newdate{date}{05}{04}{2020}
\date{\displaydate{date}}
\author{Halil Eralp Koçaş}

\titlegraphic{\includegraphics[scale=0.2]{../KU-Beamer-Template/style/images/kolt_logo.png}}

\setbeamercovered{invisible} % transparent

\begin{document}
    \maketitle

    \frame{\frametitle{Agenda}\tableofcontents}

    \section{Python Modules and Packages}

    \begin{frame}{Python Modules}
        \LARGE
        \begin{itemize}
            \item Modules are files that contains Python statements and definitions
            \pause
            \item Modules help you to break down large programs into small manageable and organized files
            \pause
            \item Modules provides reusability of a code
            \pause
            \item Most used functions can be defined in Modules and can be used in other programs without copying the definitions into every program
            \pause
            \item To use definitions in modules, modules are needed to be \textbf{imported}
        \end{itemize}
    \end{frame}
     
    \begin{frame}{Python Packages}
        \LARGE
        \begin{itemize}
            \item Packages can be thought as directories in which multiple modules are present
            \pause
            \item Packages are organized hierarchically
            \pause
            \item Packages can contain subpackages as well as regular modules
            \pause
            \item All packages are modules but not all modules are packages
            
        \end{itemize}
    \end{frame}

    \begin{frame}
        \frametitle{Importing Modules}
        \LARGE
        \pause
        \textbf{Import a module:}\\
        \pause
        \texttt{\textbf{import} module\_name}  \# all definitions \\
        \pause
        \texttt{\textbf{import} module\_name \textbf{as} name}  \# module can be used with "name" \\
         \pause
        \texttt{\textbf{from} module\_name \textbf{import} func1}  \# only specified names, func1 \\
        \pause
        \texttt{\textbf{from} module\_name \textbf{import} func1 \textbf{as} function}  \# func1 will be used by calling "function" \\
        \pause
        \texttt{\textbf{from} module\_name \textbf{import} *}  \# all names in module \\
    \end{frame}

    \section{Package Management}

    \begin{frame}
        \frametitle{Python Package Index (PyPI)}
        \pause
        \huge
        Repository of software for the Python programming language.
        \pause
        \begin{itemize}
            \item 23,000+ Python3 packages.
            \pause
            \item If you want a package, PyPI probably has it. 
        \end{itemize}
        \pause
        Visit \href{https://pypi.org/}{\underline{\textit{pypi.org}}} to explore packages.
    \end{frame}

    \begin{frame}
        \frametitle{pip}
        \LARGE
        \pause
        \begin{itemize}
        \item Recommended tool for installing Python packages.
        \pause
        \item \textbf{\texttt{pip}} is already installed with modern Python distributions.
        \pause
        \item Try \texttt{pip -V} on your command line/terminal(\texttt{pip3 -V} for Macs). 
        \end{itemize}
        \pause
        \texttt{\$ pip -V}\\
        \texttt{pip 20.0.2 from --PATH\_TO\_PIP-- (python 3.5)}\\
        \pause
        \texttt{\$ python -m pip -V}\\
        \texttt{pip 20.0.2 from --PATH\_TO\_PIP-- (python version)}\\
    \end{frame}

    \begin{frame}
        \frametitle{Common pip commands}
        \LARGE
        \pause
        \textbf{Install a package:}\\
        \pause
        \texttt{\$ pip \textbf{install} package\_name}  \# latest version \\
        \pause
        \texttt{\$ pip \textbf{install} package\_name==1.0.1}  \# specific version \\
        \pause
        \texttt{\$ pip \textbf{install} package\_name>=1.0.1}  \# minimum version \\
        \pause        
        \textbf{Uninstall a package:}\\
        \pause        
        \texttt{\$ pip \textbf{uninstall} package\_name}\\
        \pause
        \textbf{Update a package:}\\
        \pause
        \texttt{\$ pip \textbf{install --upgrade} package\_name}\\
        \pause
        \textbf{Search PyPI for matches:}\\
        \pause
        \texttt{\$ pip \textbf{search} query}
    \end{frame}

    \section{Virtual Environments}

    \begin{frame}{Virtual Environments}
        \LARGE
        \pause
        A \textit{virtual environment} is an \textbf<4->{isolated} Python environment that contains the Python interpreter, installed \textbf<4->{libraries} and scripts.\\
        \pause
        \textbf{Why} do we need them?\\
        \pause
        \pause
        What happens if two different programs use the same library?
        \begin{itemize}
            \pause
            \item We might want to use different versions of the same library.
            \pause
            \item Updating a library for \textbf{Program A} can harm another \textbf{Program B}. (\textit{Breaking Changes})
            \pause
            \item We want \textbf{isolation} between programs.
        \end{itemize}
    \end{frame}

    \begin{frame}{Creating Virtual Environments}
        \LARGE
        \pause
        In Python 3.6+, the recommended way to create a virtual environment is using \textbf{venv} package, which is included in the standard installation (similar to \textbf{pip}).\\
        \pause
        \textbf{Creating a virtual environment:}\\
        \texttt{\$ \textbf{python -m venv} /path/to/new/virtualenv}\\
        \pause
        \textbf{Activating a virtual environment:}\\
        \textbf{cd}(Change directory) to virtual environment folder.\\
        In Windows: \texttt{\$ \textbf{Scripts\textbackslash activate}}\\
        In Mac/Linux: \texttt{\$ \textbf{source bin/activate}}\\
        \pause
        \textbf{Deactivating a virtual environment:}\\
        \texttt{\$ \textbf{deactivate}}
    \end{frame}
    
\end{document}