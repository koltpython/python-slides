% !TeX root = ./Lecture3.tex

\input{../KU-Beamer-Template/style/template.tex}
\usepackage{../KU-Beamer-Template/style/koc} 
\usepackage{minted}
\usepackage{upquote}
\usepackage{graphicx}

\title{KOLT Python} 
\subtitle{Branching, While Loops, Turtle Graphics \& Strings} 
\newdate{date}{10}{02}{2020}
\date{\displaydate{date}}
\author{Necla Mutlu}

% pdflatex --shell-escape lecture3.tex & pdflatex --shell-escape lecture3.tex

\titlegraphic{\includegraphics[scale=0.18]{../KU-Beamer-Template/style/images/kolt_logo.png}}

\setbeamercovered{invisible} % transparent

\begin{document}
    \maketitle

    \frame{\frametitle{Agenda}\tableofcontents}

    \section{Recap}

        \begin{frame}{Branching}
            \vspace{-3mm}
            \begin{columns}
                \column{0.5\textwidth}
                \inputminted[firstline=1, lastline=4, frame=single,framesep=2pt]{python3}{code-examples/branching.py}
                \inputminted[firstline=6, lastline=13, frame=single,framesep=2pt]{python3}{code-examples/branching.py}
                \column{0.5\textwidth}
                \inputminted[firstline=15, lastline=27, frame=single,framesep=2pt]{python3}{code-examples/branching.py}
            \end{columns}
            \begin{itemize}
                \item \texttt{<condition>} has a \textbf{\texttt{bool}} value (\texttt{True} or \texttt{False})
                \item Which expressions will be evaluated in which conditions?
            \end{itemize}
        \end{frame}

        \begin{frame}{Branching Example}
            \LARGE
            \inputminted[frame=single,framesep=2pt]{python3}{code-examples/branching_example.py}
        \end{frame}

        \begin{frame}{Branching Example}
            \LARGE
            \inputminted[frame=single,framesep=2pt]{python3}{code-examples/branching_example2.py}
        \end{frame}

        \begin{frame}{Comparison Operators}
            \LARGE
            \begin{columns}
                \begin{column}{0.4\textwidth}
                    \begin{itemize}
                        \item \texttt{<}: Strictly less than
                        \item \texttt{<=}: Less than or equal
                        \item \texttt{>}: Strictly greater than
                        \item \texttt{>=}: Greater than or equal
                        \item \texttt{==}: Equal
                        \item \texttt{!=}: Not equal
                    \end{itemize}
                \end{column}
                \begin{column}{0.6\textwidth}
                    \inputminted[frame=single,framesep=2pt]{python3}{code-examples/comparison.py}
                \end{column}
            \end{columns}
        \end{frame}

        \begin{frame}{\texttt{bool} Operators}
         \LARGE
            How to represent logical operations in Python? (and, or, not)
            \begin{table}[]
              \resizebox{0.45\textwidth}{!}{
              \begin{tabular}{|c|c|c|c|c|}
                \hline
                \textcolor{koc}{\textbf{A}} &\textcolor{koc}{\textbf{B}} & \textcolor{koc}{\textbf{A or B}} & \textcolor{koc}{\textbf{A and B}} & \textcolor{koc}{\textbf{not A}}\\ \hline
                \textbf{\texttt{True}}  & \textbf{\texttt{True}}  & \textbf{\texttt{True}}  & \textbf{\texttt{True}} & \textbf{\texttt{False}} \\ \hline
                \textbf{\texttt{True}} & \textbf{\texttt{False}}  & \textbf{\texttt{True}} & \textbf{\texttt{False}} & \textbf{\texttt{False}} \\ \hline
                \textbf{\texttt{False}} & \textbf{\texttt{True}}  & \textbf{\texttt{True}} & \textbf{\texttt{False}} & \textbf{\texttt{True}} \\ \hline
                \textbf{\texttt{False}} & \textbf{\texttt{False}} & \textbf{\texttt{False}} & \textbf{\texttt{False}} & \textbf{\texttt{True}} \\ \hline
                \end{tabular}}
            \end{table}
            \begin{columns}
                \column{0.25\textwidth}
                    \begin{itemize}
                        \item \textbf{\texttt{and}}
                        \item \textbf{\texttt{or}}
                        \item \textbf{\texttt{not}}
                    \end{itemize}
                \column{0.75\textwidth}
                   \texttt{True or False and False} $\Rightarrow$ 
                   \textbf{\texttt{True}} \\
                   \Huge
                   \textbf{WHY?}
             \end{columns}
        
        \end{frame}

        \section{Strings}
        \begin{frame}{Strings}
            \pause
            \LARGE
            \textbf{\texttt{my\_string = \textquotesingle abcde\textquotesingle}}
            \pause
            \begin{center}
                \huge
                \begin{table}[]
                    \begingroup
                    \setlength{\tabcolsep}{1pt} % Default value: 6pt
                    \begin{tabular}{rcccccl}
                     & {\color[HTML]{c4122e} 0}  & {\color[HTML]{c4122e} 1}  & {\color[HTML]{c4122e} 2}  & {\color[HTML]{c4122e} 3}\  & {\color[HTML]{c4122e} 4} &  \\
                     \textquotesingle & a  & b  & c  & d  & e & \textquotesingle \\
                     \pause
                    & {\color[HTML]{A0A0A0} -5} & {\color[HTML]{A0A0A0} -4} & {\color[HTML]{A0A0A0} -3} & {\color[HTML]{A0A0A0} -2} & {\color[HTML]{A0A0A0} -1} & 
                    \end{tabular}
                    \endgroup
                    \end{table}
            \end{center}

            \pause
            \textbf{\texttt{print(my\_string[2])}}\pause $\Rightarrow$ prints c \\
            \pause
            \textbf{\texttt{print(my\_string[-2])}}\pause $\Rightarrow$ prints d
            
        \end{frame}

        \begin{frame}{Indexing \& Slicing}
            \Large
            \pause
            Access specific characters using \textbf{indexing}, i.e, \texttt{\textbf{[index]}}\\
            \pause
            \textbf{Slice} strings by using \texttt{\textbf{[start:stop:step]}}
            \pause
            \inputminted[frame=single,framesep=2pt]{python3}{code-examples/string_index.py}
            \pause 
            \inputminted[frame=single,framesep=2pt]{python3}{code-examples/string_index2.py}
        \end{frame}


        \begin{frame}{String Operations}
            \large
            \pause
            \inputminted[frame=single,framesep=2pt, lastline=4]{python3}{code-examples/string_operations.py}
            \pause
            \inputminted[frame=single,framesep=2pt, firstline=6]{python3}{code-examples/string_operations.py}
            \pause
            \LARGE
            \texttt{\textbf{str1 + str2}} $\Rightarrow$ \textbf{Concatenate} \texttt{str1} and \texttt{str2}\\
            \pause
            \texttt{\textbf{str1 *}} $n$ $\Rightarrow$  Repeate \texttt{str1} $n$ times.
        \end{frame}

        \begin{frame}{Example: Evil Laughter}
             \url{https://github.com/koltpython/python-slides/blob/master/Lecture3/evil_laughter.md}
             \newline
             \newline
             \pause
             \centering
            \includegraphics[height=0.6\textheight]{images/evil_laugh.jpg}
           
        \end{frame}

    \section{While Loops}
        \begin{frame}{While Loops}
            \pause
            Repeat some \texttt{<expression>}s \underline{as long as} a \texttt{<condition>} is \texttt{True}.
            \pause
            \begin{columns}
                \begin{column}{0.45\textwidth}
                    \inputminted[frame=single,framesep=2pt]{python3}{code-examples/while1.py}
                \end{column}
            \pause 
                \begin{column}{0.45\textwidth}
                    \inputminted[frame=single,framesep=2pt]{python3}{code-examples/while2.py}
                \end{column} 
            \end{columns}
            \pause
            \LARGE
            \inputminted[frame=single,framesep=2pt]{python3}{code-examples/while3.py}
            \pause
            \texttt{<condition>} is only checked \textbf{\underline{before}} each execution.
        \end{frame}

        \begin{frame}{Example: Evil Laughter (Cont.)}
            \centering
            \includegraphics[height=0.8\textheight]{images/evil_laugh.jpg}
        \end{frame}

   

    \section{Turtle}
        \begin{frame}{Turtle Module}
                \begin{center}
                    \includegraphics[width=0.3\textwidth]{images/turtle.jpg}                
                \end{center}
                \pause
                \LARGE
                a Python feature like a drawing board, which lets us command a turtle to draw all over it$\dots$\\
            \end{frame}
        \begin{frame}{Turtle Functions}
            \pause
            \LARGE
            \texttt{forward(distance)} \newline
            moves the turtle forward by the specified distance \newline
            \newline
            \pause
             \texttt{backward(distance)} \newline
             moves the turtle backward by the specified distance\newline\newline
            \pause
            \texttt{pos()} \newline returns the turtle's position\newline\newline
           
            
            
            \bigskip
            \end{frame}
        \begin{frame}{Turtle Functions}
            \LARGE
            \texttt{setpos(x,y)} \newline
            sets the turtle's position to specified x, y coordinates\newline \newline
            \pause 
            \texttt{right(angle)} \newline  
            turns the turtle right by angle units\newline\newline
            \pause
            \texttt{left(angle)} \newline  
            turns the turtle left by angle units\newline\newline
           
            \bigskip
            \end{frame}
        \begin{frame}{Turtle Functions}
         \LARGE
            \texttt{setx(x)} \newline
            sets the turtle's x coordinate to specified x\newline\newline
            \pause
            \texttt{sety(y)} \newline
            sets the turtle's y coordinate to specified y\newline\newline
            \pause
            \texttt{xcor()}   \newline   
            returns the turtle's x coordinate\newline\newline
            \bigskip
            \end{frame}
        \begin{frame}{Turtle Functions}
        \LARGE
        \texttt{ycor()} \newline
            sets the turtle's y coordinate to specified y\newline\newline
            \pause
            \texttt{pendown()} \newline
            pulls the pen down – drawing when moving.\newline\newline
            \pause
            \texttt{penup()}\newline
            pulls the pen down – drawing when moving.\newline\newline
            
        \end{frame}
    \begin{frame}{Let's draw!}
            \large
            Make the turtle draw 9 squares side by side.
            \pause
             \newline Decompose the task! What about writing a function that draws only one square?
            \pause
            \inputminted[frame=single,framesep=2pt, lastline=10]{python3}{code-examples/draw.py}
            \pause
        \end{frame}
     \begin{frame}{Let's draw!}
            \large
            How will we make it draw 9 squares by using this function and while loops?
            \pause
            \inputminted[frame=single,framesep=2pt, firstline=12]{python3}{code-examples/draw.py}
            \pause
        \end{frame}
       
   
\end{document}
