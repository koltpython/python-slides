% !TeX root = ./Lecture4.tex

\input{../KU-Beamer-Template/style/template.tex}
\usepackage{../KU-Beamer-Template/style/koc} 
\usepackage{minted}
\usepackage{upquote}
\usepackage{graphicx}

\title{KOLT Python} 
\subtitle{Lists \& For Loops} 
\newdate{date}{18}{02}{2020}
\date{\displaydate{date}}
\author{Fırat Tamur}

% pdflatex --shell-escape lecture4.tex & pdflatex --shell-escape lecture4.tex

\titlegraphic{\includegraphics[scale=0.18]{../KU-Beamer-Template/style/images/kolt_logo.png}}

\setbeamercovered{invisible} % transparent
\makeatletter
\let\@@magyar@captionfix\relax
\makeatother
\begin{document}
    \maketitle

    \frame{\frametitle{Agenda}\tableofcontents}

    \section{Recap}

        \begin{frame}{Strings}
            \LARGE
            \textbf{\texttt{my\_string = \textquotesingle abcde\textquotesingle}}
            \begin{center}
                \huge
                \begin{table}[]
                    \begingroup
                    \setlength{\tabcolsep}{1pt} % Default value: 6pt
                    \begin{tabular}{rcccccl}
                    & {\color[HTML]{c4122e} 0}  & {\color[HTML]{c4122e} 1}  & {\color[HTML]{c4122e} 2}  & {\color[HTML]{c4122e} 3}\  & {\color[HTML]{c4122e} 4} &  \\
                    \textquotesingle & a  & b  & c  & d  & e & \textquotesingle \\
                    & {\color[HTML]{A0A0A0} -5} & {\color[HTML]{A0A0A0} -4} & {\color[HTML]{A0A0A0} -3} & {\color[HTML]{A0A0A0} -2} & {\color[HTML]{A0A0A0} -1} & 
                    \end{tabular}
                    \endgroup
                    \end{table}
            \end{center}

            \textbf{\texttt{print(my\_string[2])}} $\Rightarrow$ prints c \\
            \textbf{\texttt{print(my\_string[-2])}} $\Rightarrow$ prints d
        \end{frame}

        \begin{frame}{Indexing \& Slicing}
            \Large
            Access specific characters using \textbf{indexing}, i.e, \texttt{\textbf{[index]}}\\
            \textbf{Slice} strings by using \texttt{\textbf{[start:stop:step]}}
            \inputminted[frame=single,framesep=2pt]{python3}{../Lecture3/code-examples/string_index.py}
            \inputminted[frame=single,framesep=2pt]{python3}{../Lecture3/code-examples/string_index2.py}
        \end{frame}

        \begin{frame}{String Operations}
            \large
            \inputminted[frame=single,framesep=2pt, lastline=4]{python3}{../Lecture3/code-examples/string_operations.py}
            \inputminted[frame=single,framesep=2pt, firstline=6]{python3}{../Lecture3/code-examples/string_operations.py}
            \LARGE
            \texttt{\textbf{str1 + str2}} $\Rightarrow$ \textbf{Concatenate} \texttt{str1} and \texttt{str2}\\
            \texttt{\textbf{str1 *}} $n$ $\Rightarrow$  Repeate \texttt{str1} $n$ times.
        \end{frame}

        \begin{frame}{While Loops}
            Repeat some \texttt{<expression>}s \underline{as long as} a \texttt{<condition>} is \texttt{True}.
            \begin{columns}
                \begin{column}{0.45\textwidth}
                    \inputminted[frame=single,framesep=2pt]{python3}{../Lecture3/code-examples/while1.py}
                \end{column}
                \begin{column}{0.45\textwidth}
                    \inputminted[frame=single,framesep=2pt]{python3}{../Lecture3/code-examples/while2.py}
                \end{column} 
            \end{columns}
            \LARGE
            \inputminted[frame=single,framesep=2pt]{python3}{../Lecture3/code-examples/while3.py}
            \texttt{<condition>} is only checked \textbf{\underline{before}} each execution.
        \end{frame}

    \section{Lists}
        \begin{frame}{Lists}
            \begin{center}
                \includegraphics[width=0.6\textwidth]{../Lecture1/images/box_many.jpg}                
            \end{center}
            \LARGE
            Imagine variables, but with limitless capacity$\dots$\\
            \textbf{\texttt{sunnyside = [\textquotesingle Mr. Potato Head\textquotesingle, \textquotesingle Hamm\textquotesingle,
            \textquotesingle Buzz Lightyear\textquotesingle, \textquotesingle Slinky Dog\textquotesingle]}}
        \end{frame}

        \begin{frame}{Lists}
            \huge
            \inputminted[frame=single,framesep=2pt]{python3}{../Lecture3/code-examples/intro_lists.py}
            \inputminted[frame=single,framesep=2pt]{python3}{../Lecture3/code-examples/mixed_list.py}
        \end{frame}

        \begin{frame}{Accessing Elements}
            \LARGE
            \texttt{\textbf{values = [1, \textquotesingle hello\textquotesingle, None, [3], True]}}\\
            \begin{center}
                \huge
                \begin{table}[]
                    \begingroup
                    \setlength{\tabcolsep}{2pt} % Default value: 6pt
                    \begin{tabular}{rcccccl}
                     & {\color[HTML]{c4122e} 0}  & {\color[HTML]{c4122e} 1}  & {\color[HTML]{c4122e} 2}  & {\color[HTML]{c4122e} 3}\  & {\color[HTML]{c4122e} 4} &  \\
                     $[$ & 1,  & \textquotesingle hello\textquotesingle,  & \texttt{None},  & \texttt{[3]},  & \texttt{True} & $]$ \\
                    & {\color[HTML]{A0A0A0} -5} & {\color[HTML]{A0A0A0} -4} & {\color[HTML]{A0A0A0} -3} & {\color[HTML]{A0A0A0} -2} & {\color[HTML]{A0A0A0} -1} & 
                    \end{tabular}
                    \endgroup
                    \end{table}
            \end{center}
            Use \textbf{indexing} to access and \textbf{update} elements inside list.\\
            \textbf{\texttt{print(values[2])}}\\
            \textbf{\texttt{values[2] = \textquotesingle new value\textquotesingle}} 

        \end{frame}

        \begin{frame}{Adding New Elements}
            \LARGE
            Append elements at the end of a list by \textbf{append()}
            \bigskip
            \normalsize
            \inputminted[frame=single,framesep=2pt]{python3}{../Lecture3/code-examples/append_list.py}
            \inputminted[frame=single,framesep=2pt]{python3}{../Lecture3/code-examples/append_list2.py}
        \end{frame}
        
        \begin{frame}{Inspecting List Elements}
            \LARGE
            Slice lists by using \texttt{\textbf{[start:stop:step]}}
            \normalsize
            \begin{columns}
                \begin{column}{0.35\textwidth}
                \inputminted[frame=single,framesep=2pt]{python3}{../Lecture3/code-examples/slicing.py} 
                \end{column}
                \begin{column}{0.65\textwidth}
                 \inputminted[frame=single,framesep=2pt]{python3}{../Lecture3/code-examples/slicing2.py}
                \end{column} 
             \end{columns}
        \end{frame}

        \begin{frame}{Inspecting List Elements}
            \LARGE
            \inputminted[frame=single,framesep=2pt]{python3}{../Lecture3/code-examples/slicing3.py}
            \inputminted[frame=single,framesep=2pt]{python3}{../Lecture3/code-examples/slicing4.py}
        \end{frame}
   
    
        \begin{frame}{Removing An Element}
            \pause
            \LARGE
            Remove elements in a list by \textbf{remove()}
            \pause              
            \bigskip
            \normalsize
            \inputminted[frame=single,framesep=2pt]{python3}{../Lecture3/code-examples/remove.py}
            \pause
            \LARGE
            How to avoid \texttt{ValueError}? \pause (Hint: \textbf{Branching})
        \end{frame}

        \begin{frame}{in Operator}
            \pause
            \LARGE
            Search an operand in the specified sequence by using \textbf{in}
            \pause
            \bigskip
            \inputminted[frame=single,framesep=2pt]{python3}{../Lecture3/code-examples/in_operator.py}
            \pause
            \begin{itemize}
                \item Works with both lists and strings
                \pause
                \item Works with ranges
            \end{itemize}
        \end{frame}

        \begin{frame}{len() Function}
            \pause
            \LARGE
            \texttt{len()} is an operator to determine the size of lists, strings, etc.
            \pause
            \bigskip
            \inputminted[frame=single,framesep=2pt]{python3}{../Lecture3/code-examples/length.py}
        \end{frame}

        \begin{frame}{List Mutation}
            \Large
            \textbf{\texttt{list.append(x)}}: Append x to end of the sequence\\
            \textbf{\texttt{list.insert(i, x)}}: Insert x to index i\\
            \textbf{\texttt{list.pop(i=-1)}}: Remove and return element at index i\\
            \textbf{\texttt{list.remove(x)}}: Remove first occurrence of x\\
            \textbf{\texttt{list.extend(iterable)}}: Add all elements in iterable to end of list\\
            \textbf{\texttt{list[i] = new\_value}}: Update value of index i with new value\\
            \textbf{\texttt{list[basic\_slice] = iterable}}: Change elements in basic slice with elements in iterable, sizes can be different: \texttt{numbers[:] = []}\\
            \textbf{\texttt{list[extended\_slice] = iterable}}: Change elements in extended slice with elements in iterable 1-1, sizes must be equal.\\
        \end{frame}

        \begin{frame}{Some Other List Operations}
            \Large
            \textbf{\texttt{in}} operator: Check whether an element is in list.\\
            \texttt{3 in numbers} $\Rightarrow$ \texttt{True}\\
            \textbf{\texttt{len(list)}}: Returns the length of list(and other collections).\\
            \textbf{\texttt{list.index(value, start=0, stop=len(list))}}:\\
            Return first index of value.\\
            \textbf{\texttt{list.count(value)}}: Count number of occurrences of value.\\
            \textbf{\texttt{list.reverse()}}: Reverse the list (in-place)\\
            \textbf{\texttt{list.sort()}}: Sort list elements (in-place)\\
            \\ 
            For more, type \texttt{help(list)} in your interactive interpreter.
        \end{frame}

        \begin{frame}{KAHOOT TIME}
            \LARGE{To play visit the \href{https://kahoot.it/}{https://kahoot.it/}}
            
        \end{frame}

        \section{For Loops}
      
        \begin{frame}{range() Function}
            \pause
            \LARGE
            \texttt{range(\textbf{start, stop, step})} is a function to create ranges
            \bigskip
            \inputminted[frame=single,framesep=2pt]{python3}{../Lecture3/code-examples/range.py}
        \end{frame}
        
        \begin{frame}{For Loops}
            \pause
            \begin{columns}
                \begin{column}{0.5\textwidth}
                    \inputminted[frame=single,framesep=2pt]{python3}{../Lecture3/code-examples/for1.py}
                \end{column}
               \pause 
                \begin{column}{0.5\textwidth}
                    \inputminted[frame=single,framesep=2pt]{python3}{../Lecture3/code-examples/for2.py}
                    \pause
                    \inputminted[frame=single,framesep=2pt]{python3}{../Lecture3/code-examples/for3.py}
                    \pause
                    \inputminted[frame=single,framesep=2pt]{python3}{../Lecture3/code-examples/for4.py}
                \end{column} 
            \end{columns}
        \end{frame}
        
        \begin{frame}{Example: Mail Sender}
            \pause
            \LARGE
            Fill out the attendance form:
            \newline
            \newline
            \href{https://tiny.cc/koltpython}{\underline{\textit{tiny.cc/koltpython}}}
            \newline
            \newline
            Keyword: Gentlemen
        \end{frame}

        \begin{frame}{Break, Continue \& Pass}
            \begin{columns}
                \begin{column}{0.5\textwidth}
                    \textbf{\texttt{break}} immediately terminates the closest loop
                    \bigskip  
                    \inputminted[frame=single,framesep=2pt]{python3}{../Lecture3/code-examples/break1.py}
                    \pause
                    \inputminted[frame=single,framesep=2pt]{python3}{../Lecture3/code-examples/break2.py}
                \end{column}
               \pause 
                \begin{column}{0.5\textwidth}
                    \textbf{\texttt{continue}} skips to the next iteration of the loop
                    \bigskip  
                    \inputminted[frame=single,framesep=2pt]{python3}{../Lecture3/code-examples/continue1.py}
                    \pause
                    \inputminted[frame=single,framesep=2pt]{python3}{../Lecture3/code-examples/continue2.py}
                \end{column} 
            \end{columns}
        \end{frame}
        
        \begin{frame}{Break, Continue \& Pass}
            \pause
            \LARGE
            \textbf{\texttt{pass}} does not have an effect
            \bigskip  
            \inputminted[frame=single,framesep=2pt]{python3}{../Lecture3/code-examples/pass.py}
            \inputminted[frame=single,framesep=2pt,lastline=2]{python3}{code-examples/pass_class.py}
            \pause
            \begin{itemize}
                \item Loops, conditional statements, functions, classes etc. cannot be empty 
            \end{itemize}
        \end{frame}
\end{document}